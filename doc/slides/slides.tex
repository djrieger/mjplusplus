%% LaTeX-Beamer template for KIT design
%% by Erik Burger, Christian Hammer
%% title picture by Klaus Krogmann
%%
%% version 2.1
%%
%% mostly compatible to KIT corporate design v2.0
%% http://intranet.kit.edu/gestaltungsrichtlinien.php
%%
%% Problems, bugs and comments to
%% burger@kit.edu

\documentclass[18pt]{beamer}

%% SLIDE FORMAT

% use 'beamerthemekit' for standard 4:3 ratio
% for widescreen slides (16:9), use 'beamerthemekitwide'

\usepackage{templates/beamerthemekit}
% \usepackage{templates/beamerthemekitwide}

% DEUTSCHE SPRACHE EINBINDEN
\usepackage[utf8]{inputenc}

% Deutsche Ausgabe anpassen
\usepackage[T1]{fontenc}

%% TITLE PICTURE

% if a custom picture is to be used on the title page, copy it into the 'logos'
% directory, in the line below, replace 'mypicture' with the 
% filename (without extension) and uncomment the following line
% (picture proportions: 63 : 20 for standard, 169 : 40 for wide
% *.eps format if you use latex+dvips+ps2pdf, 
% *.jpg/*.png/*.pdf if you use pdflatex)

%\titleimage{mypicture}

%% TITLE LOGO

% for a custom logo on the front page, copy your file into the 'logos'
% directory, insert the filename in the line below and uncomment it

\titlelogo{mj++logo}

% (*.eps format if you use latex+dvips+ps2pdf,
% *.jpg/*.png/*.pdf if you use pdflatex)

%% TikZ INTEGRATION

% use these packages for PCM symbols and UML classes
% \usepackage{templates/tikzkit}
% \usepackage{templates/tikzuml}

% the presentation starts here

\title[]{mj++}
\subtitle{Compilerpraktikum WS 2014/15}
\author{Gruppe 5: mj++}
\date{10. Februar 2015}

\institute{Institut für Programmstrukturen und Datenorganistation}

% Bibliography

%\usepackage[citestyle=authoryear,bibstyle=numeric,hyperref,backend=biber]{biblatex}
%\addbibresource{templates/example.bib}
%\bibhang1em

\usepackage{courier}
\usepackage{listings}
\lstset{frame=single,basicstyle=\small\ttfamily,language=Java}

\usepackage{xcolor}
\definecolor{lightgray}{gray}{0.9}
\definecolor{kitblue}{RGB}{227, 232, 242}
\newcommand{\code}[1]{\colorbox{lightgray}{\texttt{\upshape #1}}}
\newcommand{\token}[1]{\colorbox{kitblue}{\texttt{\upshape #1}}}

\begin{document}

\selectlanguage{ngerman}

\begin{frame}
\titlepage
\end{frame}

\begin{frame}{Outline}
\tableofcontents
\end{frame}

\section{MiniJava}

\begin{frame}
    \frametitle{MiniJava}
    \begin{itemize}
        \item MiniJava $\subsetneq$ Java
        \item Typen: \code{int}, \code{boolean}, Arrays, Objekte
            \begin{itemize}
                \item \emph{nicht}: \code{String}, \code{double}, \code{float}, \ldots
            \end{itemize}
        \item keine Vererbung, Interfaces, Exceptions, \ldots
        \item nur \code{public}-Member
        \item keine \code{static}-Methoden
        \item nicht alle Operatoren (z.B.\ \code{++})
    \end{itemize}
    \vskip 1cm
    \begin{center}
        Ziel: MiniJava auf x86\_64 kompilieren
    \end{center}
\end{frame}

\section{Lexikalische Analyse}

\begin{frame}
    \frametitle{Lexikalische Analyse}
    \framesubtitle{Überblick}
    \lstinputlisting{foo.mj}
    \token{class}, \token{Foo}, \token{\{}, \token{public}, \token{int}, \token{bar}, \token{(}, \token{)}, \token{\{}, \ldots \\
            \texttt{KEYWORD\_CLASS}, \texttt{IDENT}, \texttt{L\_BRACE}, \ldots
\end{frame}

\begin{frame}
    \frametitle{Lexikalische Analyse}
    \framesubtitle{Ansätze}
    \begin{itemize}
        \item Push-Interface mit Callbacks
        \item Pull-Ansatz mit \code{get\_next\_token()}
    \end{itemize}
    Endlicher Automat
    \begin{itemize}
        \item Ausprogrammieren (z.B.\ mit \code{switch})
        \item Tabelle
    \end{itemize}
    \vskip 1cm
    \begin{center}
        mj++: Pull-Ansatz und Tabelle
    \end{center}
\end{frame}

\begin{frame}
    \frametitle{Lexikalische Analyse}
    \framesubtitle{Graph}
    \begin{itemize}
        \item Graph zusammenklicken
        \item daraus Tabelle \code{transition} bauen
        \item aktueller Zustand \code{i}, Eingabezeichen \code{c} \\[3pt] \hskip .5cm $\Rightarrow$ nächster Zustand \code{transitions[i][c]}
    \end{itemize}
\end{frame}


\begin{frame}
    \frametitle{Lexikalische Analyse}
    \framesubtitle{Erfahrung}
    \begin{itemize}
        \item ohne Keyword-Unterscheidung sehr schnell
        \item Keyword-Unterscheidung aufwändig\\[1pt]
            \hskip .5cm $\Rightarrow$ dedizierte Keyword-Tabelle
        \item für Operatoren kleine Maps
        \item Graph-Dumping sinnvoll
    \end{itemize}
\end{frame}


\section{Syntaktische Analyse}
\begin{frame}
\end{frame}

\section{mj++ Besonderheiten}
\begin{frame}
\end{frame}

\end{document}
