\begin{frame}
    \frametitle{Lexikalische Analyse}
    \framesubtitle{Überblick}
    \lstinputlisting{foo.mj}
    \token{class}, \token{Foo}, \token{\{}, \token{public}, \token{int}, \token{bar}, \token{(}, \token{)}, \token{\{}, \ldots \\
            \texttt{KEYWORD\_CLASS}, \texttt{IDENT}, \texttt{L\_BRACE}, \ldots
\end{frame}

\begin{frame}
    \frametitle{Lexikalische Analyse}
    \framesubtitle{Ansätze}
    \begin{itemize}
        \item Push-Interface mit Callbacks
        \item Pull-Ansatz mit \code{get\_next\_token()}
    \end{itemize}
    Endlicher Automat
    \begin{itemize}
        \item Ausprogrammieren (z.B.\ mit \code{switch})
        \item Tabelle
    \end{itemize}
    \vskip 1cm
    \begin{center}
        mj++: Pull-Ansatz und Tabelle
    \end{center}
\end{frame}

\begin{frame}
    \frametitle{Lexikalische Analyse}
    \framesubtitle{Graph}
    \begin{itemize}
        \item Graph zusammenklicken
        \item daraus Tabelle \code{transition} bauen
        \item aktueller Zustand \code{i}, Eingabezeichen \code{c} \\[3pt] \hskip .5cm $\Rightarrow$ nächster Zustand \code{transitions[i][c]}
    \end{itemize}
\end{frame}


\begin{frame}
    \frametitle{Lexikalische Analyse}
    \framesubtitle{Erfahrung}
    \begin{itemize}
        \item ohne Keyword-Unterscheidung sehr schnell
        \item Keyword-Unterscheidung aufwändig\\[1pt]
            \hskip .5cm $\Rightarrow$ dedizierte Keyword-Tabelle
        \item für Operatoren kleine Maps
        \item Graph-Dumping sinnvoll
    \end{itemize}
\end{frame}
