\begin{frame}
    \frametitle{Lexikalische Analyse}
    \framesubtitle{Überblick}
    \lstinputlisting{foo.mj}
    \pause
     \begin{tabular}{lllllll}
         \visible<2->{\token{class}} & \visible<3->{\token{Foo}} & \visible<4->{\token{\{}} & \visible<5->{\token{public}} & \visible<6->{\token{int}} & \visible<7->{\token{bar}} & \visible<8->{\token{(}}\\
         \only<2>{\texttt{KEYWORD\_CLASS}} & \only<3>{\texttt{IDENT}} & \only<4>{\texttt{L\_BRACE}} & \only<5>{\texttt{KEYWORD\_PUBLIC}} & \only<6>{\texttt{KEYWORD\_INT}} & \only<7>{\texttt{IDENT}} & \only<8>{\texttt{L\_PAREN}}
     \end{tabular}
\end{frame}

\begin{frame}[t]
    \frametitle{Lexikalische Analyse}
    \framesubtitle{Ansätze}
    \vskip .4cm
    Endlicher Automat
    \begin{itemize}
        \item<2-> Ausprogrammieren
            \only<2>{
                \lstinputlisting[basicstyle=\ttfamily\scriptsize,language=C]{lexer-switch-case.code}
            }
        \item<3-> Tabelle
            \only<4->{
            \begin{tikzpicture}[overlay]
                \node at (.8,.2) {\includegraphics[angle=15,origin=c,width=1.5cm,keepaspectratio=true]{logos/mj++logo.png}};
            \end{tikzpicture}
        }
            \lstinputlisting[basicstyle=\ttfamily\scriptsize,language=C]{lexer-table.code}
        \item<5-> woher kommt die Tabelle?
            \begin{itemize}
                \item<6-> Graph zusammenklicken $\Rightarrow$ daraus Tabelle bauen
            \end{itemize}
    \end{itemize}
\end{frame}

\begin{frame}
    \frametitle{Lexikalische Analyse}
    hier Bild vom Graph + Dumping sinnvoll
\end{frame}

\begin{frame}[t]
    \frametitle{Lexikalische Analyse}
    \vskip 1.8cm
    Zwei Ansätze:
    \begin{itemize}
        \item<1-> Push-Interface
            \only<1>{
                \\[3pt] \code{void handle\_token(token t);}
                \\[3pt] \code{void lex(F callback);}
            }
        \item<2-> Pull-Ansatz
            \only<2>{
                \\[3pt] \code{token get\_next\_token();}
            }
    \end{itemize}
\end{frame}

\begin{frame}
    \frametitle{Lexikalische Analyse}
    \framesubtitle{Erfahrung}
    \begin{itemize}
        \item ohne Keyword-Unterscheidung sehr schnell
        \item Keyword-Unterscheidung aufwändig\\[1pt]
            \hskip .5cm $\Rightarrow$ dedizierte Keyword-Tabelle
        \item für Operatoren kleine Maps
        \item Graph-Dumping sinnvoll
    \end{itemize}
\end{frame}
